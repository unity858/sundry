\documentclass{article}
\usepackage[labelsBySect]{seto}
\title{seto.sty Sample\\(a nyanya.sty variant)}
\author{Neal Yan heheheh}
\date{May 2021}

\begin{document}

\maketitle
A demonstration of the features of seto.sty.
\section*{Package information}
There is an option called \texttt{labelsBySect}, which labels problems, theorems, etc by section.
\toc

\section{This is a section header}
\subsection{Subsection appearance}

The command for (sub)(sub)section has not changed, although its appearance has: \fakecommand{(sub)(sub)section}.

\subsubsection{Just in case!}
\section{Math commands}

A few math commands have been defined, such as \fakecommand{symsum} and \fakecommand{cycsum}, which appear as $\symsum$ and $\cycsum$, respectively\footnote{sorry dennis lol}.\\
Some other ones defined include \fakecommand{deg}(changed from `deg' to `$\deg$'), and \fakecommand{pow}(power of a point).

\subsection{Between bars}
\fakecommand{abs/norm/sqbrack/parenth} are some of the provided delimiters. Evalat is also rather straightforward:\\
\fakecommand{evalat[upper]\{lower\}}$\Rightarrow\evalat[\text{first argument}]{\text{second argument}}$;

\subsection{Floors/ceilings}
Due to personal need\footnote{Ross quiz, anyone?}, I've defined floor brackets because ``\fakecommand{lfloor \dots \textbackslash  rfloor}" was too tedious to repeat.  Although \fakecommand{floor/ceil} already uses \fakecommand{left \textbackslash right}, the sizes can also be adjusted manually by prefixing \texttt{big}, \texttt{bigg}, or \texttt{Bigg} before \texttt{floor\{...\}}/\texttt{ceil\{...\}}:
\begin{center}
\begin{tabular}{c|c|c|c|c}
    \emph{Command} & \fakecommand{floor} & \fakecommand{bigfloor} & \fakecommand{biggfloor} & \fakecommand{Biggfloor}\\\hline
    \emph{Appearance} & $\floor{\enskip}$ & $\bigfloor{\enskip}$ & $\biggfloor{\enskip}$ & $\Biggfloor{\enskip}$ \\
\end{tabular}  
\end{center}

\subsection{Sets}

I also found typing \fakecommand{mathbb\{R\}} for $\R$ too tedious to repeat, so I shortened it to \fakecommand{R}. Similar commands exist for $\N,\Z,\Q$, and $\C$:
\begin{center}
\begin{tabular}{c|c|c|c|c|c}
\emph{Command}& \fakecommand{N} & \fakecommand{Z} & \fakecommand{Q} & \fakecommand{R} & \fakecommand{C}\\\hline
\emph{Appearance}& $\N$ & $\Z$ & $\Q$ & $\R$ & $\C$
\end{tabular}  
\end{center}

\section{Fonts/text formatting}

On this sty I changed up the fonts:
\begin{itemize}
    \item Title/section/quotes font is {\Alegreya Alegreya};
    \item Sans font used is {\lato Lato}.
    \item Body text font is EB Garamond.
\end{itemize}
\fakecommand{emph}'s appearance has changed as well, \emph{like so}.
For itemize I'll change up the item icons later, using TikZ probably.\\
\newpage
\quote{Quotes are called using `\textbackslash quote\{text\}\{author\}', although this type looks more suitable at the top of a page.}{seto.sty}
\section{Fancy stuff}
\problem{The appearance of a problem/example. Who needs environments, thus these boxes are defined as commands instead of environments. This one is called by \fakecommand{problem/example[<label>]\{<text>\}}. When the first argument is omitted, the labeling parentheses will also disappear.}
\theorem{Theorems and definitions are this color. (\fakecommand{theorem/definition\{\}})}
\lemma{Lemmas/claims are this color. (\fakecommand{lemma/claim\{\}})}
\pro{I'm also too lazy to add a frame/box to enclose proofs, so whatever. (\fakecommand{pro\{\}})}

\remark{I used a [hopefully] better color theme(for anyone who cares, it's Big Sur). Oh, and remarks/notes are called by \fakecommand{remark/note\{...\}}.}

Oh, and to remove the labeling numbers, use a shortened version of the original command:
\begin{center}
\begin{tabular}{c|c|c|c|c}
    \emph{Original}& \fakecommand{problem} & \fakecommand{example}&\fakecommand{theorem}&\fakecommand{definition}\\\hline
    \emph{Shortened} & \fakecommand{prob} & \fakecommand{exmp}&\fakecommand{theo}&\fakecommand{defn}
\end{tabular}  
\end{center}

The other commands do not have abbreviated versions.

\emph{Table of Contents: }it finally worked!!!! Command is \fakecommand{toc}.

\section{Exercises}
\noindent
\exercise The bolded problem number and label is called by \fakecommand{exercise}. It takes an optional input as a label.
\exercise[sample label] Another exercise. 
\solution Solution works similarly, and the square at the end of this dummy solution is \fakecommand{solqed}.\solqed

\inlinequote{Inline quotes( \fakecommand{inlinequote\{\}\{\}}) look like this.}{seto.sty}

I think that's about it! A bit more features by now\dots
\end{document}
