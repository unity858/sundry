\documentclass{article}
\usepackage{nyanya}
\title{nyanya1.sty Sample}
\author{Neal Yan}
\date{April 2021\\ Presenting the most garish color theme to ever exist}

\begin{document}

\maketitle

\toc

\section{This is a section header}
\subsection{Subsection appearance}

The command for (sub)(sub)section has not changed, although its appearance has: \courier{\textbackslash(sub)(sub)section}.

\subsubsection{Just in case you need these.}

There is a slight possibility of the necessity of this command, \courier{\textbackslash subsubsection}, I guess.

\section{Math commands}

A few math commands have been defined, such as \courier{\textbackslash sym} and \courier{\textbackslash cyc}, which appear as $\sym$ and $\cyc$, respectively\footnote{Inspired by shootinglucky in his \emph{lucky.sty}}.

\subsection{Between bars}

After noticing the lack of absolute value bars and the evaluated-at thing, the following were defined: 

\courier{\textbackslash abs\{...\}}$\Rightarrow\abs{\dots}$;

\courier{\textbackslash evalat[upper number]\{lower number\}}$\Rightarrow\evalat[\text{first argument}]{\text{second argument}}$.

\subsection{Floors/ceilings}
Due to personal need\footnote{Ross quiz, anyone?}, I've defined floor brackets because ``\courier{\textbackslash lfloor \dots \textbackslash  rfloor}" was too tedious to repeat.  Furthermore, the brackets can be of specified size. The basic commands are \courier{\textbackslash floor} and \courier{\textbackslash ceil}. They may be made larger by prefixing \courier{big}, \courier{bigg}, or \courier{Bigg} before \courier{floor\{...\}}/\courier{ceil\{...\}}, which already include \courier{\textbackslash left} and \courier{\textbackslash right} and will change size as according to what is within the floor/ceil bars:
\begin{center}
\begin{tabular}{c|c|c|c|c}
    \emph{Command} & \courier{\textbackslash floor} & \courier{\textbackslash bigfloor} & \courier{\textbackslash biggfloor} & \courier{\textbackslash Biggfloor}\\\hline
    \emph{Appearance} & $\floor{\enskip}$ & $\bigfloor{\enskip}$ & $\biggfloor{\enskip}$ & $\Biggfloor{\enskip}$ \\
\end{tabular}  
\end{center}

\subsection{Sets}

I also found typing \courier{\textbackslash mathbb\{R\}} for $\R$ too tedious to repeat, so I shortened it to \courier{\textbackslash R}. Similar commands exist for $\N,\Z,\Q$, and $\C$:
\begin{center}
\begin{tabular}{c|c|c|c|c|c}
\emph{Command}& \courier{\textbackslash N} & \courier{\textbackslash Z} & \courier{\textbackslash Q} & \courier{\textbackslash R} & \courier{\textbackslash C}\\\hline
\emph{Appearance}& $\N$ & $\Z$ & $\Q$ & $\R$ & $\C$
\end{tabular}  
\end{center}

\section{Fonts/text formatting}

Two commonly used fonts are provided: \courier{Courier New} and \latinmodernsans{Latin Modern Sans}, which are accessed by \courier{\textbackslash courier\{...\}} and \courier{\textbackslash latinmodernsans\{...\}}, respectively.
To \emph{emphasize} a word, the command \courier{\textbackslash emph} is still used.

As for the environments \courier{enumerate}, \courier{itemize}, and \courier{enumitem}, I'm too lazy to change those, lel.

Also, the command \courier{\textbackslash solution}(which compiles as "\textit{Solution. }", because \courier{\textbackslash textit\{Solution. \}} is also rather long.

\section{Fancy stuff}
\problem{The appearance of a problem. It is called by the command 

\courier{\textbackslash problem[optional argument for source]\{problem text\}}. When the first argument is omitted, it will display ``source unknown" in place of a source.}
\example{The appearance of an example. Its use and function is identical to that of \courier{problem}, but its command is differently-named and its appearance is different: \courier{\textbackslash example[...]\{...\}}.}
\theorem{Theorems and definitions are this color. (\courier{\textbackslash theorem})}
\definition{About the same. (\courier{\textbackslash definition})}
\lemma{The boxes are all relatively similar, except for the different colors. Lemma boxes may be called by \courier{\textbackslash lemma\{lemma text\}}.}
\claim{This box is similar to \courier{\textbackslash lemma}, but color-coded differently. The command for this box is \courier{\textbackslash claim\{...\}}.}
\pro{I do not think proofs need their own box... at least not now. The command for proofs like this one is \courier{\textbackslash pro\{proof text\}}.}

\remark{The tasteless color theme is from Nyan Cat(excluding the rainbows), judge me. Oh, and remarks are called by \courier{\textbackslash remark\{...\}}. I'm also too lazy to choose different colors for \courier{lemma} and \courier{claim}.}

Oh, and to remove the labeling numbers, use a shortened version of the original command:
\begin{center}
\begin{tabular}{c|c|c|c|c|c}
    \emph{Original}& \courier{\textbackslash (sub)(sub)section} & \courier{\textbackslash problem} & \courier{\textbackslash example}&\courier{\textbackslash theorem}&\courier{definition}\\\hline
    \emph{Shortened}& \courier{\textbackslash (sub)(sub)sctn} & \courier{\textbackslash prob} & \courier{\textbackslash exmp}&\courier{\textbackslash theo}&\courier{\textbackslash defn}
\end{tabular}  
\end{center}

The other commands do not have abbreviated versions.

\emph{Table of Contents: }it finally worked!!!! Command is \courier{\textbackslash toc}.

\section{Exercises}
\noindent
\exercise The bolded problem number and label is called by \courier{\textbackslash exercise}. (who needs inputs lol)
\exercise Another exercise. (No, jk, this isn't from NARML...)
\\[4pt]
That's about all nyanya1.sty has to it!\\

\remark{This is my first .sty, and in future stys I will improve the color theme and change up the fonts.}
\end{document}
