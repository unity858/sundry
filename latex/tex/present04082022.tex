\documentclass{xuan}
\title{Dirichlet Potpourri}
\author{Youth EUCLID}
\date\today
\def\ntsum{\sum_{d\mid n}}
\def\infsum#1{\sum_{#1\ge1}}
\def\L#1{L(s,\chi_{#1})}
\begin{document}
\maketitle
\begin{frame}{Series}
\begin{pt}[Dirichlet's arith prog theorem]
For any $a,b$ with $a\perp b$, there are an infinite number of primes congruent to $a\pmod{b}$.
\end{pt}
Recall the Riemann zeta:
\[\zeta(s)=\sum_{n\ge1}1/n^s.\]
\[\Rightarrow A(s)=\sum_{n\ge1}a_n/n^s\]
\end{frame}
\begin{frame}{Properties}
Vanilla convolution: $a_n,b_n\to$
\[\sum_{k\in[0,n]}a_kb_{n-k}\] 
Convolve $A,B\Rightarrow$
\[(A*B)(s)=\sum_{n\ge1}n^{-s}\sum_{d\mid n}a_db_{n/d}\]
\end{frame}
\begin{frame}{Notable series}
$\mu$: $\sum_{d\mid n}\mu(d)=[n=1]$
\[M(s)\zeta(s)=1\Rightarrow M=1/\zeta.\]
$\sigma(n)=\sum_{d\mid n}d;\to$
\[\sum_{n\ge1}\sigma(n)/n^s=\zeta(s)\sum_{n\ge1}n/n^s=\zeta(s)\zeta(s-1).\]
$d(n)=\sum_{d\mid n}1\Rightarrow$
\[\sum_{n\ge1}d(n)/n^s=\zeta(s)^2.\]
$\sum_{d\mid n}\phi(d)=n\Rightarrow$
\[\parenth{\sum_{n\ge1}\phi(n)/n^s}\zeta(s)=\sum_{n\ge1}n/n^s=\zeta(s-1).\]
\[\Rightarrow \sum_{n\ge1}\phi(n)/n^s=\zeta(s-1)/\zeta(s).\]
\end{frame}
\begin{frame}{More functions	}
Von Mangoldt lambda:
\[\Lambda(n)=
\begin{cases}
\ln p (n=\text{ prime power});\\
0\text{ otherwise.	}
\end{cases}\]
\[\Lambda\to-\zeta '/\zeta.\]
\end{frame}
\begin{frame}{Bounding zeta}
\begin{pt}[Zeta bounds]
$\forall s>1$,
$\zeta$:
\[\zeta(s)=1/(s-1)+O(1).\]
$\zeta'$:
\[\zeta'(s)=-1/(s-1)^2+O(1).\]
$\zeta'/\zeta$:
\[(-\zeta'/\zeta) (s)=1/(s-1)+O(1).\]
\end{pt}
\begin{proof}
As earlier set real $s>1$.
Bound term-by-term:
\[\int_{n-1}^nu^{-s}du\le n^{-s}\le\int_{n-1}^nu^{-s}du+1.\]
Sum $\forall n\ge1$:
\[\zeta(s)\ge \int_1^\infty du/u^s+ smth\Rightarrow 1/(s-1)+O(1)\]
For $\zeta'$, do $d/ds$ and observe that all sides monotonic (decreasing wrt x).
\end{proof}
\end{frame}
\begin{frame}{Characters}
A {\itshape Dirichlet character mod n} is an NT function $\chi$ such that
\begin{itemize}
\item $\chi$ totally multiplicative;
\item $\chi$ determined by mod n; that is,
\[a\equiv b\pmod{n}\iff \chi(a)=\chi(b)\]
\item Hence $\chi\in{0,\pm1}$.
\item $a\not\perp n\iff\chi(a)=0$
\end{itemize}
Basic char mod n: only 0 or 1, no negatives.\\
L-series:
\[L(s,\chi)=\infsum\chi(n)/n^s.\]
Euler's factoring: for totally multiplicative $f$, we have 
\[F(s)=\prod_p(1+f(p)/p^s).\]
\end{frame}
\begin{frame}{Mod 4 example}
\begin{pt}[Lemma]
There are $\phi(n)$ characters mod $n$.
\end{pt}
Proof left as an exercise for the reader.
\\[4pt]
There are two chars mod 4, say the basic/main/principal one, $\chi_0$, and the `other' one, say $\chi_1$.
\begin{itemize}[label=*]
\item $(\chi_0+\chi_1)/2=[n\equiv 1\pmod4];$
\item $(\chi_0-\chi_1)/2=[n\equiv 3\pmod4];$
\end{itemize}
Needed result:
\[\ln\L{}=\infsum\frac{\Lambda(n)}{\ln n}\chi(n)/n^s.\]
\end{frame}
\begin{frame}{Proof cont'd}
\begin{pt}[Claim]
$\L{0}$ diverges as $s\to1$.
\end{pt}
\begin{proof}
By a known result, $\sum_p\chi(p)/p^s=\ln\L{}+O(1)$, which implies the result.
\end{proof}
\begin{pt}[Claim]
$\L1$ is finite.
\end{pt}
\begin{proof}
Simply apply alternating series test or whatever.
\end{proof}
Taking linear combos we get both cases of Dirichlet in one go for $\mod4$.
\end{frame}
\begin{frame}{Generalisation to general n}
\begin{pt}[Lemma]
For a nonprincipal char $\chi\pmod{n}$, $\sum_{k=1}^n\chi(k)=0.$
\end{pt}
\begin{proof}
Simply observe that all such $\chi$ are totally multiplicative and there is at least one $-1$ val. This implies the result.
\end{proof}
Hence $\L{}=O(1)$ for all nonprinicpal $\chi$ and $s>0$ (not 1).
Finally, it takes a bit more work to show that
\[L(1,\chi)=\infty\]
for principal $\chi\pmod{n}$.
\\[4pt]
After taking linear combos and using the $O(1)$ bound from earlier, we finally get Dirichlet's arith prog theorem :0
\end{frame} 	
\begin{frame}
\begin{pt}[The end?]
Have a good sleep everyone :D
\end{pt}
\end{frame}
\end{document}